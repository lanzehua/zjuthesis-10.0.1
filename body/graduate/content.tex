\chapter{绪论}

% 本模板根据浙江大学研究生院编写的《浙江大学研究生学位论文编写规则》~\cite{zjugradthesisrules},
% 在原有的 zjuthesis 模板~\cite{zjuthesis}基础上开发而来。

% 本模板的本科生版本\cite{zjuthesisrules}得到了浙江大学本科生院老师的支持与审核,


\section{研究背景}

文化兴则国运兴,文化强则民族强。文化是国家的灵魂,民族的血液。2024年7月,党的二十届三中全会胜利完会,全会审议通过的《中共中央关于进一步全面深化改革、推进中国式现代化的决定》-(1)中特别指出,要优化文化产品供给机制,激发全民族文化创新创造活力,这其中有两个着力点,就是强化创作导向、借力技术创新。为了更好更牢地以文化为重要支点推动经济高质量发展,不断实现文化与经济交融互动、融合发展,我们需要积极推进以文兴业、科技赋能,大力发展以“文化+创意”“文化+科技”为主要特征的文化创意产业,成为有效应对发展挑战、培育新的经济增长点的突破口-(2)。尽管文创领域发展一片良好,但现阶段还是存在一些问题:
\
(1)文创产品的质量不够高,难以满足人们日益增长的审美需求。
(2)文创开发的效率较低下,对于某一难以在短时间内产出多种相关的创作成果,难以在快速波动的市场中抓住商业机会。
为了更好地聚焦于两个着力点,强化文化数字产品的创作能力,有效提升我国文化创作水平,我们可以利用风格迁移技术。风格迁移技术是一种强大的技术,通过往特定场景中引入风格化图像的风格特征并融合处理,创造出新的风格化的艺术场景。其中二维图像风格迁移可以用在广告、设计和媒体等领域,通过改变图像的外观和装饰,使其更加引人注目和独特。三维风格迁移是将一个场景的风格提取并且应用到另一个场景的三维模型上,可以为虚拟现实、游戏开发和电影制作等领域注入新动能,推动文化产业的创新创优,丰富人民的文化生活。
经过多年的发展,风格迁移技术得到了长足进步,但是还是存在以下问题-(3):
(1)图像的风格纹理特征不够细致。现有的图像的任意风格迁移算法不能在生成无伪影的高质量图像的同时充分兼顾如颜色,笔触,色调,纹理等艺术特征,导致。现有的内容和风格特征之间对齐的方法还需要进一步完善,以保障更高质量的特征混合。同时现有的风格特征使用VGG-(4)编码器进行提取,它更适合分类任务的特征处理,而在关注颜色,笔触、纹理等艺术特征方面还可以进一步优化。
(2)3D场景风格迁移技术的性能权衡问题:现有的3D场景迁移模型多少在以下方面有一些不足:多视角一致性不够、对任意一个新的风格图片都需要重新进行风格化训练、训练推理速度较慢等、在面对更多的约束和挑战下,场景的风格化质量不佳等。有的模型结构过于复杂对设备要求较高,而且将它们应用到现实中充满各种物体的3D环境中存在较大不足。


\begin{itemize}
    \item 删除根目录的 ``.latexmkrc'' 文件,否则编译失败且不报任何错误
    \item 字体有版权所以本模板不能附带字体,请务必手动上传字体文件,并在各个专业模板下手动指定字体。
        具体方法参照 GitHub 主页的说明。
    \item 当前的Overleaf默认使用TexLive 2017进行编译,但一些伪粗体复制乱码的问题需要TexLive 2019版本来解决。
        所以各位同学可以在Overleaf上编写论文时务必切换到TexLive 2019或更新版本来编译,以免产生查重相关问题。
        具体说明参照 GitHub 主页。
\end{itemize}


\section{节标题}

我们可以用includegraphics来插入现有的jpg等格式的图片,
如\autoref{fig:zju-logo}所示。

\begin{figure}[htbp]
    \centering
    \includegraphics[width=.3\linewidth]{logo/zju}
    \caption{\label{fig:zju-logo}浙江大学LOGO}
\end{figure}


\subsection{小节标题}


\par 如\autoref{tab:sample}所示,这是一张自动调节列宽的表格。

\begin{table}[htbp]
    \caption{\label{tab:sample}自动调节列宽的表格}
    \begin{tabularx}{\linewidth}{c|X<{\centering}}
        \hline
        第一列 & 第二列 \\ \hline
        xxx & xxx \\ \hline
        xxx & xxx \\ \hline
        xxx & xxx \\ \hline
    \end{tabularx}
\end{table}


\par 如\autoref{equ:sample},这是一个公式

\begin{equation}
    \label{equ:sample}
    A=\overbrace{(a+b+c)+\underbrace{i(d+e+f)}_{\text{虚数}}}^{\text{复数}}
\end{equation}

\chapter{另一章}


\begin{figure}[htbp]
    \centering
    \includegraphics[width=.3\linewidth]{example-image-a}
    \caption{\label{fig:fig-placeholder}图片占位符}
\end{figure}

\chapter{再一章}

\par 如\autoref{alg:sample},这是一个算法

\begin{algorithm}[H]
    \begin{algorithmic} % enter the algorithmic environment
        \REQUIRE $n \geq 0 \vee x \neq 0$
        \ENSURE $y = x^n$
        \STATE $y \Leftarrow 1$
        \IF{$n < 0$}
            \STATE $X \Leftarrow 1 / x$
            \STATE $N \Leftarrow -n$
        \ELSE
            \STATE $X \Leftarrow x$
            \STATE $N \Leftarrow n$
        \ENDIF
        \WHILE{$N \neq 0$}
            \IF{$N$ is even}
                \STATE $X \Leftarrow X \times X$
                \STATE $N \Leftarrow N / 2$
            \ELSE[$N$ is odd]
                \STATE $y \Leftarrow y \times X$
                \STATE $N \Leftarrow N - 1$
            \ENDIF
        \ENDWHILE
    \end{algorithmic}
    \caption{\label{alg:sample}算法样例}
\end{algorithm}