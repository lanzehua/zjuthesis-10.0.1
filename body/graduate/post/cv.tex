\cleardoublepage
\chapternonum{攻读硕士学位期间主要的研究成果}

\noindent \textbf{专利和论文}
\par [1]基于大模型风格先验知识的风格迁移方法、计算机设备、可读存储介质和程序产品[P]. 中国专利:CN 118014821 A,2024.05.10. 发明人顺序为导师第一、本人第二。
\par [2]三维场景风格迁移方法、三维场景风格迁移系统和计算机设备[P]. 中国专利:CN 117274042 A,2023.12.22. 发明人顺序为导师第一、本人第二。
\par [3]DuDoINet: Dual-Domain Implicit Network for Multi-Modality MR Image Arbitrary-scale Super-Resolution[C].//In Proceedings of the 31st ACM International Conference on Multimedia (pp. 7335-7344).本人为第四作者(若不算导师和导师组成员为第二作者)。
\par [4]Rethinking Video Deblurring with Wavelet-Aware Dynamic Transformer and Diffusion Model[C].// In: European Conference on Computer Vision. Springer, Cham, 2025. p. 421-437. 本人为第三作者。
\par [5]Rethinking Multi-Contrast MRI Super-Resolution: Rectangle-Window Cross-Attention Transformer and Arbitrary-Scale Upsampling[C].//In: Proceedings of the IEEE/CVF International Conference on Computer Vision. 2023. p. 21230-21240.本人为第四作者(若不算导师和导师组成员为第三作者)。
\par [6]Self-Reference Image Super-Resolution via Pre-trained Diffusion Large Model and Window Adjustable Transformer[C].//In: Proceedings of the 31st ACM International Conference on Multimedia. 2023. p. 7981-7992.本人为第四作者(若不算导师和导师组成员为第二作者)。
\newline
\textbf{参与的项目}

\par [1] 国家自然科学基金项目:基于解耦学习的艺术图像修复关键技术研究(No:62172365)。
\par [2] 浙江省科技计划项目:文物保护与交易流通关键技术及产品研发(No:2023C03199)。
\par [3] 宁波市科技计划项目:宁波“大运河-海丝之路”文旅融合虚拟云平台关键技术攻关(No:2023Z137)。
\par [4] 浙江大学教育部脑与脑机融合前沿科学中心:类脑艺术创作智能的模型与方法研究。
\par [5] 浙江省科技计划项目:数据知识双轮驱动的跨模态智能检索与生成平台及应用(No:2024C01110)。


