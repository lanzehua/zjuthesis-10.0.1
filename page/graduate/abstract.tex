\cleardoublepage
\chapternonum{摘要}

\setlength{\parindent}{2em} 
近年来,风格迁移技术,作为一种将特定艺术风格应用到给定图像或场景内容中的技术,已经成为数字艺术创作的重要工具。
风格迁移用途广泛,能应用到2D 图像上,也可以应用到 3D 场景上。现有的风格迁移技术在 2D 和 3D 方面的应用都存在一些问题,
在 2D 场景中,风格的迁移不够精细、不能很好地实现局部和全局风格的平衡。
而现有的 3D 场景风格迁移技术,均不同程度地在多视角一致性、训练推理速度、生成质量等方面存在一些问题。
针对上述问题,本文设计了一个新的 2D 图像任意风格迁移算法,并提出了一个支持不同风格迁移方法的零样本 3D 场景快速风格迁移方法。具体工作主要包括以下两个方面:
\newline \indent(1)提出一种基于风格一致性实例归一化和对比学习的2D图像任意风格迁移算法。其中风格一致性实例归一化 (SCIN)模块用来捕获远程和全局的风格相关性。
这可以将内容特征与风格特征对齐;基于实例的对比学习(ICL)方法用来学习风格化到风格化的关系并消除伪影;
并且我们提出了一个新的感知编码器(PE),它可以捕获风格信息,避免模型过多关注风格图像的显著分类特征。   
\newline \indent(2)基于多分辨率哈希网格和可变3D高斯点的零样本三维场景风格迁移方法。
该方法使用3DGS的表示作为场景的建模表示方法,以保障风格化的一致性,和利用它的实时渲染特性实现快速风格化。
采用球谐函数的增量来表示高斯点的颜色变化量,并且使用多分辨率哈希网格来实现大量点的增量查询并缓解直接使用神经网络做变换导致的大显存占用问题。
此外我们还依据高斯点的致密化轮次对其进行分层,只选用外层的点进行位置增量计算,减少了部分计算量的同时也实现了视觉效果的优化。
\newline
{\addfontfeatures{FakeBold=2.0}\textbf{关键词:}}
任意风格迁移;3D高斯飞溅;零样本推理;多分辨率哈希网格

\cleardoublepage
\chapternonum{Abstract}
In recent years, style transfer techniques, as a way of applying a specific artistic style to a specified image or scene content, has become an important tool for digital art creation. 
Style transfer can be applied to both 2D images and 3D scenes. There are some problems in the application of existing style transfer techniques in 2D and 3D. 
In 2D scenes, the style transfer is not fine enough to achieve a good balance between local and global style. 
However, the existing 3D scene style transfer techniques have some problems in multi-view consistency, training and inference speed, and generation quality to varying degrees. 
To address the above problems, we design a new arbitrary style transfer algorithm for 2D images and propose a fast zero-shot style transfer method for 3D scenes that supports different style transfer methods. The specific work mainly includes the following two aspects:
\newline \indent(1) An arbitrary style transfer algorithm for 2D images based on style-content instance normalization and contrastive learning. 
The Style-Content Instance Normalization (SCIN) module is used to capture remote and global style correlations. 
This aligns content features with style features; 
Instance-based Contrastive learning (ICL) is used to learn stylization relationships and eliminate artifacts. 
And we propose a novel perceptual encoder (PE), 
which can capture style information and avoid the model focusing too much on the salient classification features of style images.
\newline \indent(2) A zero-shot 3D scene style transfer method based on multi-resolution hash grid and variable 3D Gaussian points. 
In this method, 3DGS is used as the modeling representation of the scene to ensure the consistency of the style, 
and its real-time rendering feature is used to achieve fast stylization. 
The increment of spherical harmonic function is used to represent the color change of Gaussian points, 
and the multi-resolution hash grid is used to realize the incremental query of a large number of points and alleviate the problem of large video memory usage caused by directly using neural network to transform.
In addition, Gaussian points are stratified according to their densification rounds, 
and only the points in the outer layer are selected for position incremental calculation, 
which reduces part of the calculation and realizes the optimization of visual effects.
\newline
{\addfontfeatures{FakeBold=2.0}\textbf{Keywords:}} 
Arbitrary Style Transfer; 3D Gaussian Splatting; Zero-Shot Inference; Multi-Resolution Hash Grid
